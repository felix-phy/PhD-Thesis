\chapter{Introduction}

The Standard Model (SM) of particle physics is the theory that best decribes our current understanding of fundamental particles and their interactions. It describes a broad range of phenomena and makes a plethora predictions for these, many of which have been confirmed via measurement to great degrees of accuracy \cite{PhysRevD.110.030001}. A noteable feature of the SM is the Brout-Englert-Higgs (EBH) mechanism \cite{Higgs}\cite{EnglertBrout}, which predicts the existance of a Brout-Englert-Higgs (or often simply Higgs) boson. The EBH mechanism is considered a central part of the SM as it provides a unique mechanism by which SM particles may acquire mass through their interaction with the Higgs boson. As such, the experimental discovery of a Higgs-like scalar boson in 2012 \cite{HiggsDiscovery2012CMS}\cite{HiggsDiscovery2012ATLAS} was a major milestone in particle physics. Since this discovery, a significant open question in particle physics has been whether this particle indeed behaves entirely in an SM-like way. Measuring the exact properties of the discovered scalar particle has thus been a major feature of LHC experiments such as the CMS collaboration \cite{CMS_Collaboration2022-xw}. A significant subset of these properties are the so-called Yukawa interactions between the Higgs boson and massive fermions. As can be seen in \autoref{fig:Higgs_couplings}, a number of these have previously been measured and indeed align with the values expected from the SM. However, the measurement of the Yukawa couplings of several of the lighter fermions still remain an open challenge as these couplings decrease in strength with smaller fermion masses. \\
\\
The next lightest fermion candidate for such a measurement is the charm quark. Consequentially, the study of the Yukawa-coupling between the Higgs boson and the charm quark is of significant interest \cite{PhysRevD.100.073013}. Apart from a brief discussion of the SM, this section introduces the charm-Yukawa coupling. Additionally, LHC processes may be targeted to exploit their sensitivity to the Higgs-charm Yukawa coupling with an experiment such as the CMS detector are discussed. 

\begin{figure}
    \centering
    \includegraphics[width=0.8\textwidth]{figures/Higgs_couplings.png}
    \caption{The measured coupling modifiers of the coupling between the Higgs boson and fermions as well as heavy gauge bosons as functions of fermion or gauge boson mass m$_{f/V}$, where $\nu$ is the vacuum expectation value of the Higgs field. \cite{CMS_Collaboration2022-xw} NOTE: coupling modifiers}
    \label{fig:Higgs_couplings}
\end{figure}


\section{The Standard Model of particle physics}
The SM is formulated as through the formalism of Quantum Field Theory (QFT) \cite{something here}. This is a formalism that combines concepts of classical field theory, quantum machanics as well as special relativity into a single, coherent description of fundamental particles as excitations of underlying fields that pervade space-time. In this description, SM particles fall into two categories: fermions and bosons. The former are the massive particles which may make up the matter of the universe while the latter are the force-carrying particles of the strong and electro-weak forces. The distinction between these categories is made based on the spin of the particle, which may be of either half-integer or integer respectively. \\
\\
The fermion content of the SM consists of 12 unique particles. These include six leptons, namely the electron, muon and tau as well as their respective neutrinos as well as six different quarks that are distinguished by their so-called flavour. The different quark flavours include up, down, charm, strange, bottom and top and specifies a quark's mass eigenstate as well as electric charge. These fermions are typically arranged into three generations typically depicted as

\begin{align}
        \begin{pmatrix}
            e \\
            \nu_{e}
        \end{pmatrix}
        \begin{pmatrix}
            \mu \\
            \nu_{\mu}
        \end{pmatrix}
        \begin{pmatrix}
            \tau \\
            \nu_{\tau}
        \end{pmatrix}
        \, , \,
        \begin{pmatrix}
            \, u \, \\
            \, d \,
        \end{pmatrix}
        \begin{pmatrix}
            \, c \, \\
            \, s \, 
        \end{pmatrix}
        \begin{pmatrix}
            \, t \, \\
            \, b \,
        \end{pmatrix}.
\end{align}
\\
However, there are distinct differences between the leptons and quarks. Leptons carry integer (or no) charge while quarks carry fractional charges. More importantly, while both quarks and leptons may interact via the electro-weak force, only the quarks interact via the strong force. Due to the nature of the strong force, quarks almost exlusively form compositive states called hadrons. Lastly, the existence of anti-fermions must be mentioned. These carry the exact opposite quantum numbers (.e.g charge) as their fermion counterparts, though otherwise behave similarly (take the electron and positron for instance). For simplicity, references to a fermion in this work may be understood as referencing both the fermion and anti-fermion counterpart, unless otherwise explicitly indicated. Examples of the latter are e.g. referring explicitly to electrons e$^{-}$ and positrons e$^{+}$ or charm quark c and anti-charm quark $\bar{\mathrm{c}}$ pairs. \\
\\
There exist 13 unique bosons in the SM. These include the photon $\gamma$, W$^{\pm}$ and Z which mediate the electro-weak force as well as 8 gluons $g$ that mediate the strong force. The final piece is the Higgs boson. Contrary to the force carriers, which all are spin 1, the Higgs boson is spin 0. By interacting with the Higgs boson, the massive particles of the SM acquire their mass and is thus a central element of the SM. \\
\\
Considering the introduced particles and forces, the SM has a rich and detailed phenomeology. A great example of a mathematically rigurous delineation of this can be found for example in \cite{PeskinSchroeder}. Given the focus of this work on the Yukawa coupling between the Higgs boson and charm quark, only this aspect of the SM is discussed in further detail.

\section{The Higgs-charm Yukawa coupling and the cH process}

\section{SMEFT}

