\chapter{The CMS experiment at the LHC}
The Compact Muon Solenoid (CMS) detector \cite{CMSDetector} is large, general purpose particle detector located at the Large Hadron Collider (LHC)\cite{LHC} accelerator in Geneva, Switzerland. Run by the European Oragnisation for Nuclear Research (CERN), the LHC's largest ring spans a circumference of 27km, making it the largest particle accelerator in the world. In their circular trajectory throug the beam pipe, collimated bunches of $\sim$ 10$^{11}$ protons are accelerted in both directions of the ring. At each of the four collision points, of which CMS is built around one, the trajectories of these proton bunches are crossed such that highly energetic proton-proton collisions are produced. A detector such as CMS effectively acts as a camera taking very complex snapshot of each collision. During Run 2 of the LHC, approximately 30 protons collide on average per bunch crossing with a centre of mass energy of $\sqrt{s}$ = 13 TeV. These collisions produce a plethora of particle, many of which decay to particles of varying multiplicities themselves. As such, these collision produce a complex and varied phenomenology that require a complex machine such at the CMS detector to fully capture. By capturing the information from many millions of collisions, a multitude of different statistical analyses may be performed on the captured data. This includes analyses of the Higgs boson and its properties, such as the Yukawa coupling of the charm quark. To this end, this chapter gives an overview of the CMS detector and its subsystems as well as the techniques used to reconstruct individual proton-proton collisions. 
\section{The CMS detector}
The CMS detector is designed to be able to detect a wide range of signatures and is built from a set of complementary sub-detectors. An overview of the detector may be seen in \autoref{fig:CMSDetector}. By combining data from these sub-detectors, a comprehensive reconstruction of individual proton-proton collisions, commonly referred to as an \textit{event}, may be made. The role and functioning of the individual sub-detectors is covered in this section. 
\subsection{The CMS coordinate system}
\subsection{The silicon tracker}
\subsection{The electromagnetic calorimeter}
\subsection{The hadronic calorimeter}
\subsection{The superconducting solenoid magnet}
\subsection{The muon chambers}
\subsection{The triggering system}
\section{Event reconstruction with the CMS detector}