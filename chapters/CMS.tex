\chapter{The CMS experiment at the LHC}
The Compact Muon Solenoid (CMS) detector \cite{CMSDetector} is large, general purpose particle detector located at the Large Hadron Collider (LHC)\cite{LHCDesignReport} accelerator in Geneva, Switzerland. Run by the European Oragnisation for Nuclear Research (CERN), the LHC's largest ring spans a circumference of 27km, making it the largest particle accelerator in the world. In their circular trajectory through the beam pipe, collimated bunches of $\sim$ 10$^{11}$ protons are accelerted in both directions of the ring. At each of the four collision points, of which CMS is built around one, the trajectories of these proton bunches are crossed such that highly energetic proton-proton collisions are produced. A sketch of the LHC accelerator complex can be seen in \autoref{fig:LHC}. A detector such as CMS effectively acts as a camera taking very complex snapshot of each collision. During Run 2 of the LHC, approximately 30 protons collide on average per bunch crossing with a centre of mass energy of $\sqrt{s}$ = 13 TeV. These collisions produce a plethora of particle, many of which decay to particles of varying multiplicities themselves. As such, these collision produce a complex and varied phenomenology that require a complex machine such at the CMS detector to fully capture. By capturing the information from many millions of collisions, a multitude of different statistical analyses may be performed on the captured data. This includes analyses of the Higgs boson and its properties, such as the Yukawa coupling of the charm quark. To this end, this chapter gives an overview of the CMS detector and its subsystems as well as the techniques used to reconstruct individual proton-proton collisions. 

\begin{figure}
    \centering
    \includegraphics[width=0.9\textwidth]{figures/LHC.png}
    \caption{An overview of the LHC accelerator complex \cite{LHCSketch}. Before enetering the large LHC ring, particles must pass through a number of increasingly powerful set of accelerators.}
    \label{fig:LHC}
\end{figure}

\begin{figure}
    \centering
    \includegraphics[width=0.9\textwidth]{figures/CMS.png}
    \caption{An overview of the CMS detector \cite{CMSDesignReportVol1}.}
    \label{fig:CMSDetector}
\end{figure}

\section{The CMS detector}
The CMS detector is designed to be able to detect a wide range of signatures and is built from a set of complementary sub-detectors. An overview of the detector may be seen in \autoref{fig:CMSDetector}. By combining data from these sub-detectors, a comprehensive reconstruction of individual proton-proton collisions, commonly referred to as an \textit{event}, may be made. The role and functioning of the individual sub-detectors is covered in this section. While several of the detector components have undergone changes for the current Run-3 of the LHC\cite{CMSRun3}, the configuration relevant to this work is that of Run-2. 
\subsection{The CMS coordinate system}
Due to the cylindrical nature of the CMS detector, using cylindrical coordinates to describe positions within the detector is a natural choice. Thus, the z coordinate describes the position along the beam pipe, $r$ the radius and $\phi$ the azimuthal angle, where the proton-proton collision point is taken as a the coordinate system's centre. Trajectories of particles with energy $E$ within the detector into the plane perpendicular to z may be described by the rapidity 

\begin{align}
    y = \mathrm{ln}\sqrt{\frac{E + p_{z}c}{E - p_{z}c}}.
\end{align}
\\
Small momenta in the z-direction $p_z$ give a rapidity of zero, while the radpitity tends to $\pm\infty$ for large $p_z$. However, this requires knowledge of $E$ and $p_z$, which can be difficult to measure. By assuming the particle is ultra-relativistic, as is typically the case at the LHC, it is possible to simply this description and introduce the pseudorapidity

\begin{align}
    \eta = \mathrm{ln}\left(\mathrm{tan}\left(\frac{\theta}{2}\right)\right)
\end{align}
\\
which is dependent solely on $\theta$, the polar angle. A convenient feature of the (pseudo)rapidity is that differences of (pseudo)rapidity are Lorentz invariant and thus not dependant on the initial longitudinal boost of the proton-proton system, which is a priori not known due to the varying momenta fractions of its constitutents. 
Together with the particle's transverse (to the beam axis) momentum \pt\, and mass $m$, a particles four-vector may be defined as 
\\
\begin{align}
    p = 
    \begin{pmatrix}
        m \\
        p_T \\
        \eta \\
        \phi \\ 
    \end{pmatrix} .
\end{align}
\\
The CMS detector may be broadly split into two distinct regions inward and outward of the boundary $\mid$$\eta$$\mid$ = 1.479. The inner region or \textit{barrel} consists of concentric layers around the beam pipe. The outer \textit{endcap} region consists of two caps that close off the detector at either end. In this way, the CMS detectors is designed for the best possible hermetic coverage around the collision point. 
\subsection{The silicon tracker}

\begin{figure}
    \centering
    \includegraphics[width=0.9\textwidth]{figures/CMSTracker.pdf}
    \caption{An overview of the CMS silicon tracker \cite{CMSTracker}, shown in the r-z plane after its upgrade during Run-2. The pixel detector is denoted in green while the silicon strip detector is denoted in blue and orange.}
    \label{fig:CMSTracker}
\end{figure}

The silicon tracker \cite{CMSTracker} is the innermost system of the CMS detector, situated closest to the beampipe. It is designed to track the trajectories of charged particles as they emerge from the collision point with minimal energy losses to the particles themselves. This subdetector is split into two main components, the pixel detector and silicon strip detector. A sketch of these components may be seen in \autoref{fig:CMSTracker}. \\
\\
The pixel detector is situated right around the beampipe and as of 2017 consists of four circular layers of individual silicon pxiels in the barrel region and three disk layers in the endcap region. These consist of rectangular silicon chips with a size of 100 x 150 $\mu$m$^2$. When a charged particle traverses through the active material of these chips, an electrical signal is induced that is recorded. This is typically referred to as a \textit{hit}. The small pixel size allows for position measurements with very high resolution, namely $\sim$ 10$\mu$m in the $r$$\phi$ direction and $\sim$ 20$\mu$m in the $z$ direction \cite{CMSTrackerResolution}. An important feature of the pixel detector its high radiation tolerance due to the close proximity of these modules to the beam pipe. \\
\\
Following the pixel detector is the silicon strip detector. It is composed of silicon strips of varying sizes, with increases in size at greater distances to the beam pipe due to the reduced overall particle flux they must contend with. In the barrel region, this consists of 10 layers of silicon strips, while in the endcap regions this consists of nine layers. The latter extend the coverage of the detector to $\mid$$\eta$$\mid$=2.5. \\
\\
The tracking system provides key information that is essential to the reconstruction of events. As charged particle fly through the CMS detector, their trajectories are curved due to the magnetic field generated by the solenoid magnet (see \autoref{subsection:CMSSolenoidMagnet}). By measuring the curvature of these trajectories with this system, the transverse momentum \pt \, of particles can be constructed. Additionally, the tracker plays a key role in methods used to determine the nature of hadronic particle cascades and the origin particles (quarks or gluons) from which these originate. \\
\\
\subsection{The electromagnetic calorimeter}
The second innermost subsystem is the electromagnetic calorimeter (ECAL) \cite{CMSECAL}\cite{CMSECALPerformance}. It is designed to measure the energies of electromagnetic showers initiated by photons and electrons. The ECAL is a homogenous calorimeter, consisting of over 75,000 lead tungstate crystals. These crystals scintillate as charged particles pass through them and the produced photons can be collected via photodiodes, producing an electrical signal. This signal may be evaluated to infer the energy that is deposited. Not only do the crystals scintillate but they are also extremely dense and thus are very effective in absorbing the energy of incoming electrons and photons. This allows a very compact thickness of 23cm (22cm) in the barrel (endcap) region, which corresponds to $\sim$26 ($\sim$25) radiation lengths. An additional component of the ECAl is the preshower detector. This consists of lead absorbers interlaced with scintillating layers and help to distinguish high energy photons from neutral pions. The latter decays into photon pairs which may mimic high energy photons in this part of the detector with an increased likelihood. The increased granularity of the preshower detector helps mitigate this effect. The energy resolution of the ECAl is $\sim$ 1-4\%.

\begin{figure}
    \centering
    \includegraphics[width=0.9\textwidth]{figures/CMSEcal.png}
    \caption{An overview of the CMS ECAL \cite{CMSECALFigure}, shown in the r(y)-z plane. The dashed lines denote the coverage of the barrel and endcap ECAL region as well as the preshower detector.}
    \label{fig:CMSTracker}
\end{figure}

\subsection{The hadronic calorimeter}
Following the ECAl is the hadronic calorimeter (HCAL) \cite{CMSHCAL}. It is designed to measure the presence and energy of hadrons, which typically traverse the ECAL with minor energy losses. It is the most hermetic part of the CMS detector, with a coverage out to $\mid$$\eta$$\mid$ = 5.0, in order to absord almost all collision particles. The only exceptions to this are muons which are particles that minimally deposit their energy and neutrinos, which have an interaction probability that is so low that they cannot be measured with the CMS detector at all. \\
\\
In contrast to the ECAL, the HCAL is a sampling calorimeter. This means layers of absorber are interleaved with layers of a scintillator. Different materials are used in different parts of the calorimeter, which is split into the barrel ($\mid$$\eta$$\mid$ $<$ 1.5), endcap (1.5 $<$ $\mid$$\eta$$\mid$ $<$ 3.0) and forward (3.0 $<$ $\mid$$\eta$$\mid$ $<$ 5.0) regions. Since the HCAL component inside the magnet system does not sufficiently absorb all hadronic showers, the system also extends past the magnet. Due to the sampling nature of the calorimeter, a lower nuber of respective interaction lengths and larger energy fluctuations in hadronic particle showers, the energy resolution of the HCAL is significantly poorer than the ECAL. It lies in the order of 10-30$\%$ and with a strong dependence on the energy and pseudorapidity of the initiating particles.
\subsection{The superconducting solenoid magnet}
\label{subsection:CMSSolenoidMagnet}
A key component of the CMS detector is the superconducting solenoid magnet \cite{CMSMagnet}. It is responsible for maintaining a strong 3.8 T magnetic field that homogenously permeats the barrel of the detector. A measurement of the field strength can be seen in \autoref{fig:CMSMagnet}. With its toroidal shape, the field is orientated along the z-axis and covers the 12.9m barrel region of the detector, curving the the trajectories of charged particles emerging from the interaction point in the $\phi$-direction. This allows for a measurement of the particles transverse momentum $p_{T}$, which together with the $\phi$ and $\eta$ directions fully characterise the particle's momentum vector. The magnet itself is composed of of superconducting noibium-titanum coils that are cooled to a temperature of 4.65K, at which these are superconducting. The magnet is encased by a 12,000t steel yoke that captures the magentic field that is produced outside of the solenoid. 

\begin{figure}
    \centering
    \includegraphics[width=0.9\textwidth]{figures/CMSMagnet.png}
    \caption{An overview of the magnetic flux (left) and magnetic field lines(right) inside the CMS detector, shown in the r-z plane \cite{CMSMagnetFigure}.}
    \label{fig:CMSMagnet}
\end{figure}

\subsection{The muon chambers}
The muon subdetector consists of a dedicated system of gaseous detectors \cite{CMSMuonDetector}\cite{CMSMuonDetectorPerformance}, which are placed outside of the solenoid magnet. As suggested by the CMS name, a strong focus is placed on the performance of this subdetector. This is as muons may often be produced in collisions that are of physics interest (such as in this work) and thus an emphasis is laid on detecting these with great efficiency. Due to muons being minimally ionising particles, they easily pass through the inner subdetector layers to reach the muon chambers and information from the moun chambers as well as the tracker and calorimeters may be used to identify and reconstruct them.\\
\\
Like the other subdetectors, the muon chambers are separated in a barrel ($\mid$$\eta$$\mid$ $<$ 1.2) and endcap (1.2 $<$ $\mid$$\eta$$\mid$ $<$ 2.4) region, which are composed of drift tubes and cathode strip chambers respectively. The drift tubes each consists of a gas volume containing a mixture of Argon and CO$_2$ in which a posititively charged wire is stretched through the center. When charged particles such as muons traverse these tubes, the gas is ionised. Due to the positive charge of the wire, the resulting electrons drift towards the wire producing an electrical signal. Thus the presence of muons may be determined by activation of the drift tubes. The cathode strip chambers on the other hand consist of layers of positively charged (anode) wires, which are arranged in a perpendicular fashion to a set of negatively charged (cathode) strips. Combining signals from both the wires and strips allows for a position measurement in both the R and $\phi$ direction. Both types of detector are supplemented by resistive plate chambers, which act as a trigger providing a precise timing resolution of $\sim$1ns. This makes it possible to unambiguously assign muons to individual collisions. These consist of parallel, oppositely charged plastic plates that are coated with a conductive graphite layer and are contained in a gas volume. Ionisation of the gas due to the traversal of a charged particle thus leads to an electrical signal. An overview of the spetial arrangement of these systems can be see in \autoref{fig:CMSMuonSystem}. With this system, the bulk of muons may be measured with a precise momentum resolution of $\sim$1-2\%. 

\begin{figure}
    \centering
    \includegraphics[width=0.9\textwidth]{figures/CMSMuonSystem.png}
    \caption{An overview of the CMS muon system, shown in the r-z plane \cite{CMSDesignReportVol1}. Shown are the drift tube (DT), the cathode strip chambers (CSC) and resistive plate chambers (RPC).}
    \label{fig:CMSMuonSystem}
\end{figure}

\subsection{The triggering system}
The triggering system is an essential component in manging the data output of the CMS detector \cite{CMSTrigger}. With a nominal collision rate of $\sim$40 MHz, the data rate the CMS detector provides is close to 40 TB/s. Not only is the storage of such a quantity of data unfeasible but a significant portion consists of low-energy scattering events which are not of interest. As such, the triggering system is implemented to extract a subset of events that are of physics interest. \\
\\
The trigger systems is composed of two subsystems. The first the so-called level one (L1) trigger. This is a very fast hardware-based system which reduces the event rate to $\sim$100 kHz by evaluating the presence of e.g. energetic muons or other interesting signatures such as large energy deposits in the calorimeters in an event. The total time allocated to decide whether an event should be kept is 3.2$\mu$s. Subtracting for signal propagation in the detector, the L1 system must make a decision within ~1$\mu$s. From the L1, the events are passed to a software based high-level trigger (HLT) system. This is composed of several thousand CPU cores, performing a simple reconstruction of the event signatures to make a decision whether an event should be stored. Since different analyses have different needs, a set of trigger paths are defined so that only one such path must be satisfied for an event to pass the HLT. Since the HLT is software-based, the trigger paths may be continuously updated. After the HLT, the event rate is thus reduced to $\sim$100 Hz and the passing events are permanently stored.

\section{Event reconstruction with the CMS detector}
Events that pass the triggering system are stored and reconstructed using a more complicated set of reconstruction algorithms. An overview of the reconstruction techniques for the objects relevant to this work, namely muons and jets, is given in this section. 

\subsection{Track and vertex reconstruction}
\label{subsection:TrackingAndVertexing}
Particle tracks, describing the trajectories of particles through the detector, can be obtained by leveraging information from the pixel and strip detectors of the tracker \cite{CMSTrackReco}. By determining the track of a charged particle and thus the curvature of its trajectory in the detectors magnetic field, the particle's transverse momentum \pt \, may be implicitly determined. Since track reconstruction is a computationally intensive procedure given the large number of permutations in which individual pixel or strip hits may be combined, this procedure is performed iteratively. Inititally, tracks which are easily identifiable due to e.g. their relatively high \pt \, or proximity to the interaction point are identified by matching hits in the pixel and silicon strip subdetectors and performing a fitting procedure. The hits associated with these tracks are then removed from the collection of unassociated hits. This procedure is repeated anew with looser fitting criteria so that hits that may originate from low \pt \, tracks or those with an origin displaced from the collision point, may also be associated to tracks. \\
\\
From the reconstructed tracks, common track origins or \textit{vertices} may be identified. Since several proton-proton collisions may occur in a single bunch crossing, this amounts to identifying the location of the individual collisions in an event. Tracks with a low perpendicular distance or low \textit{impact parameter} to the center of the bunch crossing and that satisfy requirements on the number of pixel and strip detector hits as well as the quality of the track fit are chosen for this purpose. These tracks are clustered using a deterministic annealing algorithm \cite{Annealing}, thus producing a set of candidate vertices with some location along the z-axis. The vertex candidate which is associated with the highest $\sum$\pt$^2$ is assigned as the primary vertex of the collision. The remaining vertex candidates are referred to as pile-up vertices. 

\subsection{The Particle Flow algorithm}
The Particle Flow (PF) algorithm \cite{ParticleFlowCMS} is used to combine information from many of the different CMS subsystems to give an improved and holistic description of an event. This includes reconstructed tracks, the energy deposits in the ECAL and HCAL as well as hits in the muon chamber system. Since different types of particles will interact with the CMS subdetector systems in unique ways, the properties of individual particles can be extrapolated from this information. These are briefly summarised in \autoref{table:PFParticleTypes}. \\
\\
\begin{table}[H]
    \centering
    \caption[]{Overview of particle signatures in the CMS detector}
    \begin{adjustbox}{width=1\textwidth}
    \label{table:PFParticleTypes}
        \begin{tabular}{l l}
        \toprule 
        \textbf{Particle} & \textbf{Signature} \\
        \midrule 
        \midrule
        Muons & Muons produce tracks in the tracker as well as the muon system \\
        & with minimal energy deposits in the calorimeters. \\
        \midrule
        Electrons & Electrons produce tracks in the tracker as well as energy deposits \\ 
        & in the ECAL with minimal deposits in the HCAL. \\
        \midrule
        Photons & Photons do not produce tracks in the tracker due to being uncharged \\
        & and deposit their energy in the ECAL. \\
        \midrule
        Charged hadrons & Charged hadrons produces tracks in the tracker, primarily depositing \\
        & their energy in the HCAL. \\
        \midrule
        Neutral hadrons & Neutral hadrons produce no tracks in the tracker, primarily depositing \\
        & their energy in the HCAL. \\ 

        \bottomrule
        \end{tabular} 

    \end{adjustbox}
\end{table}
\noindent
A visual overview of these signatures and the particle type they correspond to can be found in \autoref{fig:CMSPF}. The PF algorithm leverages exactly these properties. Initially, matched tracks in the tracker and muon systems are identified as muons and the corresponding components are removed from the event. Subsequently, matched tracks and energy deposit clusters in the ECAL are identified as electrons and the corresponding components are removed.  An isolated cluster in the ECAL with no associated track is reconstructed as a photon candidate and the corrsponding cluster is removed. This is expected to leave only charged and neutral hadrons. Clusters of energy deposits in the HCAL associated with a track are thus identified as charged hadrons. However, it frequently occurs that photons are produced in the decay of neutral hadrons. Thus, if the energy estimated from a track is considerably less than the associated cluster in the HCAL and there is a corresponding energy deposit in the ECAL, an additional photon candidate is reconstructed that is associated with the hadron. Finally, HCAL clusters with no associated track are reconstructed as neutral hadrons. \\
\\
This of course is a greatly simplified description, a more comprehensive version of which can be found in \cite{ParticleFlowCMS}. The following section describe in greater detail the reconstruction of objects relevant to this work. This includes muons, \textit{jets}, which are collimated particle showers that typically consist of a collection of reconstructed objects and missing transverse energy. 

\begin{figure}
    \centering
    \includegraphics[width=0.9\textwidth]{figures/PF.pdf}
    \caption{An transverse slice of the CMS detector, visualising the signatures that different particles produce in the different detector subsystems. \cite{ParticleFlowCMS}. }
    \label{fig:CMSPF}
\end{figure}

\subsection{Reconstruction and identification of muons}
Since muons are used to reconstruct the Higgs candidate in the cH, they represent an important element of the analysis described in this work. Using the available information from the tracker and muon system, three different approaches may be used to intially reconstruct muon tracks. 

\begin{itemize}
    \item \textbf{Standalone muon tracks}: A standalone muon track simply refers by a fit of individual hits present in the muon detector.  
    \item \textbf{Tracker muon tracks}: Tracker muon tracks are reconstructed by extrapolating tracks from the tracker to the muon detector, referred to as an \textit{inside-out} approach. If a hit in the muon detector can be matched to the extrapolated track, then these matched tracks are identified as a tracker muon track. This reduces the impact from atmospheric muons traversing the detector, which may be falsely interpreted as standalone muon tracks.
    \item \textbf{Global muon tracks}: Global muon tracks are obtained through an \textit{outside-in} appproach, by matching standalone muon tracks with tracker muon tracks through a comparison of the respective fitted track parameters. If the tracks are found to match, a combined fit of these tracks is performed. This approach reduces the impact from remnants of hadronic showers that reach the muon chambers, which may be incorrectly reconstructed as a tracker muon track. 
\end{itemize}
Naturally, there is a large overlap between global and tracker muon tracks. If two muon tracks share the same track in the tracker, then they are merged into a single object. The collection of standalone, tracker and global muons is passed to the previously introduced PF alogrithm which, by imposing additional quality requirements (see \cite{ParticleFlowCMS}) produces a set of reconstructed muon candidates. \\
\\
A useful criterium in identifying muons that originate directly from the proton-proton interaction is the relative isolation $\mathcal{I}^2$. This is defined as

\begin{align}
    \mathcal{I}^{\mu}_{\mathrm{rel}} = \bigg( \sum p_{\mathrm{T}}^{\mathrm{charged}} + \mathrm{max}\big(\sum p_{\mathrm{T}}^{\mathrm{neutral}} + \sum p_{\mathrm{T}^{\gamma}} - p_{\mathrm{T}}^{\mu\mathrm{,PU}}\big) \bigg) / p^{\mu}_{\mathrm{T}}. \label{RelativeIsoEquation}
\end{align}
Here, $\sum p_{\mathrm{T}}^{\mathrm{charged}}$ represents the salar sum of the transverse momenta of charged hadron originating from the primary vertex of the event. The quantities $\sum p_{\mathrm{T}}^{\mathrm{neutral}}$ and $\sum p_{\mathrm{T}^{\gamma}}$ represent the respective transverse momenta sums for neutral hadrons and photons. These sums are calculated by accounting from contributions within a conical volume around the muon direction. The size of a cone between two positions $i$ and $j$ is defiend as $\Delta R(i,j) = \sqrt{\Delta\eta(i,j)^2 + \Delta\phi(i,j)^2}$ and in this case the cone boundary around the muon direction is set at $\Delta R=0.4$. The contribution to the relative isolation from pile-up is estimated by subtracting $p_{\mathrm{T}}^{\mu\mathrm{,PU}} = 0.5 \sum_k p_{\mathrm{T}}^{k, \mathrm{charged}}$ in \autoref{RelativeIsoEquation}, where the sum over $k$ represents charged hadron contributions not originating from the PV. The fator 0.5 corrects for different fractions of charged and neutral particles in the cone \cite{CMSPileUpMitigation}. Lastly, $p^{\mu}_{\mathrm{T}}$ represents the transverse momentum of the muon. The relative isolation is thus a variable that quantifies the presence of energy deposits in the ECAL and HCAl around the trajectory of the muon. Since muons are expected to produce such deposits only minimally, good muon candidates are expected to be associated with small values of $\mathcal{I}^{\mu}_{\mathrm{rel}}$.\\
\\
Two sets of muon identification criteria are defined for this work:

\begin{itemize}
    \item \textbf{Loose muons}: Loose muons are PF muons reconstructed from either a global or tracker muon track where the perpendicular distance of the extraploated track to the event's primary vertex is less than 5mm in the $z$ direction and less than 2mm in the r direction. 
    \item \textbf{Tight muons}: Tight muons are loose muons which are reconstructed exclusively from a global muon track. A number of additional criteria are applied. This includes that the fit quality of the global muon track must be $\chi^2$/ndf $<$ 10 as well that the singificance of the track's 3D impact parameter SIP$_\mathrm{3D}$ = IP/$\sigma_\mathrm{IP}$ satisfies SIP$_\mathrm{3D}<$ 4. Here IP is the impact parameter or point of closest approach to the primary vertex and $\sigma_\mathrm{IP}$ is the associated uncertainty. Additionally, it is required the at least six layers with at least one pixel hits are registered in the tracker in the associated track as well as two segments hit in the muon detector. Lastly, a relative isolation requirement of $\mathcal{I}^{\mu}_{\mathrm{rel}} < 0.25$ is imposed. 
\end{itemize}
The tight muon definition is used to select muons for reconstructing Higgs candidates while the loose definition is used in the estimation of reducible backgrounds. 
\subsection{Reconstruction and identification of jets}
The quarks and gluons that are produce in pp collisions rapidly hadronise, typically producing collimated cones of particle referred to as $jets$. Details on the concept of hadronisation, which results from the nature of the strong interaction, can be found in \cite{Hadronisation}. Since the c quark of the cH process too will produce a jet, jet objects also represent an important aspect of the analysis presented in this work. \\
\\
To produce jet objects, the hadrons reconstructed by the PF algorithm must be clustered. To ensure a minimal impact of pile-up on this clustering, the contributions of pile-up are mitigated through \textit{charged hadron subtraction}. This involes the removal of charged hadron contributions in the HCAL and ECAL if these may be associated with any of the pile-up vertices produced in the collision, as described in \autoref{subsection:TrackingAndVertexing}. Once this subtraction has been performed, the remaining PF hadrons are passed to the anti-$k_\mathrm{T}$ algorithm \cite{FastJet}. The anti-$k_\mathrm{T}$ algorithm is an iterative clustering algorithm that is based on a priniciple of minimal distances between particles. The distance $d_{ij}$ between the particles $i$ and $j$ is defined as well as the distance $d_{iB}$ between particle $i$ and the beam. These are given by 

\begin{gather}
    d_{ij} = \mathrm{min}(\frac{1}{p_{\mathrm{T}, i}}, \frac{1}{p_{\mathrm{T}, j}})\frac{\Delta_{ij}^2}{R^2} \\
    d_{iB} = \frac{1}{p_\mathrm{T}, i} \\
    \Delta_{ij} = \sqrt{\Delta\mathrm{y}(i,j)^2 + \Delta\phi(i,j)^2}. 
\end{gather}
\\
Here, y is the rapidity of a particle and R is a constant parameter that determines the cone size of the clustered jets. The default choice used in CMS is R=0.4, which is also used in this work. Starting with the highest \pt \, object in the initial iteration, the distance $d_{ij}$ with the closest PF candidate $j$ is calculated. The two objects are clustered together and this process is repeated until a stopping condition $d_{ih} > d_{iB}$ is met. At this point, the jet is considered fully reconstructed and the PF candidates used in its clustering are removed for the reconstruction of subsequent jets. \\
\\
Due to the presence of detector noise, unphysical low \pt \, jets can be erroneously reconstructed. This effect can be mitigated by applying additional criteria on reconstructed jets. This includes requiring that at least two PF candidates are clustered in the jet and that the jet's energy is not solely attributed to neutral hadrons or photons. These requirements remove almost all such unphysical jets while over 99$\%$ of physical jets fulfill them \cite{CMSJetPerformance}. Additionally, a pile-up discrimination algorithm is described in \cite{CMSJetPerformance}, of which the loose working point is applied to jets with \pt\,$<50$ GeV in this work. \\
\\
A calibration of jet energies is performed after reconstruction \cite{CMSJEC} in both simulation and data. This calibration accounts for pile-up contributions in the clustering, the non-linearity of the detector response and improper reconstruction of hadrons. A number of methods are used to derive sets of correction factors. An example is the use of events with a Z boson, the \pt \, of which may be precisely reconstructed via the Z$\rightarrow\mu\mu$ decay, that a single jet recoils against. Additionally, significant discrepancies in the resolution of jets in simulation and data are obersved, with the resolution being worse in the latter than the former. This is accounted for by a smearing method, in which the resolution of jets is artifically smeared in simulation so that a better comparison to data is achieved. 

\subsection{Missing transverse momentum}
Due to the conservation of momentum, it is expected that the vectorial sum of momenta of all particles produced in a collision adds up to zero. However, this may not be the case when particles such as neutrinos are produced in a collision as these cannot be measured by the detector. As a result, it can be useful to define the missing transverse momentum as 
\begin{equation}
    p_{\mathrm{T}}^{\mathrm{miss}} = \sum^{\mathrm{PF}}_{i} p_{\mathrm{T}}^{(i)}. 
\end{equation}
The presence of significant quantities of $p_{\mathrm{T}}^{\mathrm{miss}}$ may thus be used to identify the presence of neutrinos in an event.

\section{Identification of charm quark-induced jets}
To identify the charm quark-induced jet of the cH process, one must be able to discriminate against both bottom quark as well as light quark or gluon-induced jets. This is a task often colloquially referred to as \textit{flavour tagging}, with a jet's \textit{flavour} being determined by the type of particle that inititated it. Modern flavour tagging techniques typically use machine learning to leverage key jet properties that may differentiate jets of different flavours, though this remains a challenging task. To discuss these properties, a definition of jet flavour is useful. In the context of CMS, a ghost matching procedure \cite{GhostMatching} is applied to obtain such a definition for simulated events. This involves adding information from the event simulation to the reconstructed event. Specifically, hadrons containing bottom and charm quarks are identified in the simulation and added to the list of reconstructed PF candidates, albeit with negligible momenta. With this addition of so-called \textit{ghost hadrons} the jet clustering is once again performed. Due to the neglible momenta of the ghost hadrons, the clustering procedure itself is unaffected. However, the inclusion of the ghost hadrons can be used for the following definitions:

\begin{itemize}
    \item \textbf{\cjet s}: If at least one charm ($c$) ghost hadron and no bottom ($b$) hadrons are clustered inside the jet, the jet is labelled as a $c$ jet. 
    \item \textbf{\bjet s}: If at least one $b$ ghost hadron is clustered inside the jet, the jet is labelled as a \bjet.
    \item \textbf{light jets}: If no ghost hadrons are clustered inside the jet, the jet is labelled as a light jet. Light jets may be initiated by quarks such as the up, down, or strange quark or by gluons. An additional, technical category of $pile-up jets$ exists depending on whether so-called matching criteria between reconstructed and simulated jets are fulfilled, though they are subsumed into the light jets category for the purpose of this work.
\end{itemize}
The task of identifying \cjet s is thus twofold, as \cjet s must be differentiated from both \bjet s and light (pileup) jets. This is broken down into two tasks:
\begin{enumerate}
    \item Discriminating \textit{heavy-flavour} (HF) jets consisting of \bjet s and \cjet s against light jets.
    \item Discriminating between \bjet s and \cjet s. 
\end{enumerate}

\subsection{Properties of heavy-flavour jets}
The term heavy-flavour originates from the mass of the bottom and charm quarks, which is an order of magnitude greater than the next heaviest quark, the strange quark. Similarly, $c$ and $b$ hadrons have relatively long lifetimes that allow them to travel an observable distance from the PV before decaying. The typical lifetime of a $b$ hadron of the order of  $\sim$ 1.5 ps while that of $c$ quarks ranges down to approximately an order of magnitude less \cite{PhysRevD.110.030001}. This typically results in the presence of a secondary vertex (SV) that is measurably displaced from the collision point up to a distance of 1cm in the case of energetic hadrons and is thus a key signature of HF jets. Tracks originating from the decay of a HF induced jet thus typically originate from a SV. This effect can for example be seen when looking at the significance of 2D impact parameters of $b$, $c$ and light jets, as seen in \autoref{fig:2DSIP}. \\
\\
Another feature of heavy flavour jets is the presence of leptons in the jet. This results from the relatively large branching fractions of HF hadrons into states containing leptons. These are typically low-energy and are present in about 20\% (10\%) of $b$($c$) jets, meaning the identification of a low-energy electron or muon inside a jet serves as a good indicator that a jet originates from a HF hadron. Also of significance are the relatively high masses HF hadrons exhibit in comparison to their lighter counterparts. This results in HF induced jets having a broader energy flux compared to their lighter counterparts, due to higher diffusion of momenta perpendicular to the flight direction as well as a higher hadron multiplicity resulting from the decay of the HF hadron. These features are illustrated in \autoref{fig:CMSTagging}.

\begin{figure}
    \centering
    \includegraphics[width=0.9\textwidth]{figures/2DSIP.png}
    \caption{Plots showing the significance of the 2D impact parameter of the most and second most displaced tracks in a jet \cite{CMSTagging}. As can be seen, these variables can differentiate $b$ and $c$ jets from light jets to a significant degree.}
    \label{fig:2DSIP}
\end{figure}

\begin{figure}
    \centering
    \includegraphics[width=0.9\textwidth]{figures/HFJets.png}
    \caption{An illustration highlighting the properties of HF jets \cite{CMSTagging}. The presence of a secondary vertex (SV), characterised by the impact parameter (IP) in green, as well as the presence of a lepton is highlighted.}
    \label{fig:CMSTagging}
\end{figure}

\subsection{The DeepJet algorithm}

\begin{figure}
    \centering
    \includegraphics[width=0.9\textwidth]{figures/DeepJetArch.png}
    \caption{An illustration depicting the architecture of the DeepJet neural network \cite{DeepJet}. Three individual branches separately process the charged hadrons, neutral hadrons and secondary vertex information before being passed onto a comibing, fully connected layer together with global variables.}
    \label{fig:CMSDeepJet}
\end{figure}

The \textit{DeepJet} algorithm \cite{DeepJet} is a machine learning algorithm used for jet-flavour identification in this work. It improves on previous neural network based algorithms \cite{CMSTagging} used by CMS in the Run-2 period of the LHC. A notable feature compared to earlier algorithms is its use of lower level information such as use of track, PV and SV information, as well as PF candidate and event kinematics information. An overview of the architecture employed by DeepJet can be see in \autoref{fig:CMSDeepJet}. The network is comprised of three branches that individually process neutral and charged hadrons as well as secondary vertices before this information is combined with global variables in a set of fully connected layers. The network ouput consists of six output nodes representing six individual output classes. The output value of the nodes $\mathcal{P}(b/bb/lepb/c/l/g)$ for a given jet are interpreted as the likelihood that a jet belongs to the respective class. These are defined as
\begin{itemize}
    \item{\textbf{\textit{b/bb/lepb} (\bjet s):} These three classes represent subclasses of jets originating from a b hadron. The $b$ class represents a jet originating from a single $b$ hadron, the $bb$ class originating from two $b$ hadrons and $lepb$ representing a jet originating from a $b$ hadron with the presence of a soft lepton.}
    \item{\textbf{\textit{c} (\cjet s)}: This class represents a jet originating from a $c$ hadron.}
    \item{\textbf{\textit{l, g} (light jets)}: These two classes represent light jets originating from a light quark or gluon respectively. }
\end{itemize}
From these output classes, two useful discriminators to identify \cjet s can be constructed. These are
\begin{gather}
    \mathrm{CvsL} = \frac{\mathcal{P}(c)}{\mathcal{P}(c)+\mathcal{P}(l)+\mathcal{P}(g)} \\
    \mathrm{CvsB} = \frac{\mathcal{P}(c)}{\mathcal{P}(c)+\mathcal{P}(b)+\mathcal{P}(bb)+\mathcal{P}(lepb)},
\end{gather}
representing a discrimination of \cjet s against \bjet s and light jets respectively. The performance of DeepJet with the CvsL and CvsB discriminators in simulated samples of top quark pair production can be see in \autoref{fig:DeepJetPerformance}. A comparison to the $DeepCSV$ jet-flavour identification algorithm is included, highlighting the performance gain that the DeepJet algorithm achieves.\\
\\
Since neural network based algorithms are trained on simulated samples that do not perfectly describe their data counterpart, the neural network output must be calibrated with respect to data. To calibrate the entire shape of the algorithm's output distributions the approach described in \cite{DeepJetShapeCalibration} is used. This involves targeting phase spaces enriched in $b$ jets (top quark pair production), $c$ jets (charm associated W$^{\pm}$ production) and light jets (jet associated Drell-Yan production). Using simulation, the fractions of $b$, $c$ and light-flavour jets are determined in each phase space and an interative fitting procedure, minimising differences between simulation and data is performed. This allows for the derivation of correction factors which depend on the discriminators CvsL and CvsB as well as the true flavour of a simulated jet. 

\begin{figure}
    \centering
    \includegraphics[width=0.9\textwidth]{figures/DeepJetEfficiencies.png}
    \caption{Performance of DeepJet algorithm in identifying $c$ jets against $b$ jets and light jets in simulated samples of top quark pair production, in which both top quarks decay hadronically \cite{DeepJetShapeCalibration}. The x-axis represents the efficiency with which $c$ jets are identified, while the y-axis represents mis-identification rate with respect to either $b$ jets or light jets.}
    \label{fig:DeepJetPerformance}
\end{figure}


