\chapter{The CMS experiment at the LHC}
The Compact Muon Solenoid (CMS) detector \cite{CMSDetector} is large, general purpose particle detector located at the Large Hadron Collider (LHC)\cite{LHCDesignReport} accelerator in Geneva, Switzerland. Run by the European Oragnisation for Nuclear Research (CERN), the LHC's largest ring spans a circumference of 27km, making it the largest particle accelerator in the world. In their circular trajectory through the beam pipe, collimated bunches of $\sim$ 10$^{11}$ protons are accelerted in both directions of the ring. At each of the four collision points, of which CMS is built around one, the trajectories of these proton bunches are crossed such that highly energetic proton-proton collisions are produced. A sketch of the LHC accelerator complex can be seen in \autoref{fig:LHC}. A detector such as CMS effectively acts as a camera taking very complex snapshot of each collision. During Run 2 of the LHC, approximately 30 protons collide on average per bunch crossing with a centre of mass energy of $\sqrt{s}$ = 13 TeV. These collisions produce a plethora of particle, many of which decay to particles of varying multiplicities themselves. As such, these collision produce a complex and varied phenomenology that require a complex machine such at the CMS detector to fully capture. By capturing the information from many millions of collisions, a multitude of different statistical analyses may be performed on the captured data. This includes analyses of the Higgs boson and its properties, such as the Yukawa coupling of the charm quark. To this end, this chapter gives an overview of the CMS detector and its subsystems as well as the techniques used to reconstruct individual proton-proton collisions. 

\begin{figure}
    \centering
    \includegraphics[width=0.9\textwidth]{figures/LHC.png}
    \caption{An overview of the LHC accelerator complex \cite{LHCSketch}. Before enetering the large LHC ring, particles must pass through a number of increasingly powerful set of accelerators.}
    \label{fig:LHC}
\end{figure}

\begin{figure}
    \centering
    \includegraphics[width=0.9\textwidth]{figures/CMS.png}
    \caption{An overview of the CMS detector \cite{CMSDesignReportVol1}.}
    \label{fig:CMSDetector}
\end{figure}

\section{The CMS detector}
The CMS detector is designed to be able to detect a wide range of signatures and is built from a set of complementary sub-detectors. An overview of the detector may be seen in \autoref{fig:CMSDetector}. By combining data from these sub-detectors, a comprehensive reconstruction of individual proton-proton collisions, commonly referred to as an \textit{event}, may be made. The role and functioning of the individual sub-detectors is covered in this section. While several of the detector components have undergone changes for the current Run-3 of the LHC\cite{CMSRun3}, the configuration relevant to this work is that of Run-2. 
\subsection{The CMS coordinate system}
Due to the cylindrical nature of the CMS detector, using cylindrical coordinates to describe positions within the detector is a natural choice. Thus, the z coordinate describes the position along the beam pipe, $r$ the radius and $\phi$ the azimuthal angle, where the proton-proton collision point is taken as a the coordinate system's centre. Trajectories of particles with energy $E$ within the detector into the plane perpendicular to z may be described by the rapidity 

\begin{align}
    y = \mathrm{ln}\sqrt{\frac{E + p_{z}c}{E - p_{z}c}}.
\end{align}
\\
Small momenta in the z-direction $p_z$ give a rapidity of zero, while the radpitity tends to $\pm\infty$ for large $p_z$. However, this requires knowledge of $E$ and $p_z$, which can be difficult to measure. By assuming the particle is ultra-relativistic, as is typically the case at the LHC, it is possible to simply this description and introduce the pseudorapidity

\begin{align}
    \eta = \mathrm{ln}\left(\mathrm{tan}\left(\frac{\theta}{2}\right)\right)
\end{align}
\\
which is dependent solely on $\theta$, the polar angle. A convenient feature of the (pseudo)rapidity is that differences of (pseudo)rapidity are Lorentz invariant and thus not dependant on the initial longitudinal boost of the proton-proton system, which is a priori not known due to the varying momenta fractions of its constitutents. 
Together with the particle's transverse (to the beam axis) momentum \pt\, and mass $m$, a particles four-vector may be defined as 
\\
\begin{align}
    p = 
    \begin{pmatrix}
        m \\
        p_T \\
        \eta \\
        \phi \\ 
    \end{pmatrix} .
\end{align}
\\
The CMS detector may be broadly split into two distinct regions inward and outward of the boundary $\mid$$\eta$$\mid$ = 1.479. The inner region or \textit{barrel} consists of concentric layers around the beam pipe. The outer \textit{endcap} region consists of two caps that close off the detector at either end. In this way, the CMS detectors is designed for the best possible hermetic coverage around the collision point. 
\subsection{The silicon tracker}

\begin{figure}
    \centering
    \includegraphics[width=0.9\textwidth]{figures/CMSTracker.pdf}
    \caption{An overview of the CMS silicon tracker \cite{CMSTracker}, shown in the r-z plane after its upgrade during Run-2. The pixel detector is denoted in green while the silicon strip detector is denoted in blue and orange.}
    \label{fig:CMSTracker}
\end{figure}

The silicon tracker \cite{CMSTracker} is the innermost system of the CMS detector, situated closest to the beampipe. It is designed to track the trajectories of charged particles as they emerge from the collision point with minimal energy losses to the particles themselves. This subdetector is split into two main components, the pixel detector and silicon strip detector. A sketch of these components may be seen in \autoref{fig:CMSTracker}. \\
\\
The pixel detector is situated right around the beampipe and as of 2017 consists of four circular layers of individual silicon pxiels in the barrel region and three disk layers in the endcap region. These consist of rectangular silicon chips with a size of 100 x 150 $\mu$m$^2$. When a charged particle traverses through the active material of these chips, an electrical signal is induced that is recorded. This is typically referred to as a \textit{hit}. The small pixel size allows for position measurements with very high resolution, namely $\sim$ 10$\mu$m in the $r$$\phi$ direction and $\sim$ 20$\mu$m in the $z$ direction \cite{CMSTrackerResolution}. An important feature of the pixel detector its high radiation tolerance due to the close proximity of these modules to the beam pipe. \\
\\
Following the pixel detector is the silicon strip detector. It is composed of silicon strips of varying sizes, with increases in size at greater distances to the beam pipe due to the reduced overall particle flux they must contend with. In the barrel region, this consists of 10 layers of silicon strips, while in the endcap regions this consists of nine layers. The latter extend the coverage of the detector to $\mid$$\eta$$\mid$=2.5. \\
\\
The tracking system provides key information that is essential to the reconstruction of events. As charged particle fly through the CMS detector, their trajectories are curved due to the magnetic field generated by the solenoid magnet (see \autoref{subsection:CMSSolenoidMagnet}). By measuring the curvature of these trajectories with this system, the transverse momentum \pt \, of particles can be constructed. Additionally, the tracker plays a key role in methods used to determine the nature of hadronic particle cascades and the origin particles (quarks or gluons) from which these originate. \\
\\
\subsection{The electromagnetic calorimeter}
The second innermost subsystem is the electromagnetic calorimeter (ECAL) \cite{CMSECAL}\cite{CMSECALPerformance}. It is designed to measure the energies of electromagnetic showers initiated by photons and electrons. The ECAL is a homogenous calorimeter, consisting of over 75,000 lead tungstate crystals. These crystals scintillate as charged particles pass through them and the produced photons can be collected via photodiodes, producing an electrical signal. This signal may be evaluated to infer the energy that is deposited. Not only do the crystals scintillate but they are also extremely dense and thus are very effective in absorbing the energy of incoming electrons and photons. This allows a very compact thickness of 23cm (22cm) in the barrel (endcap) region, which corresponds to $\sim$26 ($\sim$25) radiation lengths. An additional component of the ECAl is the preshower detector. This consists of lead absorbers interlaced with scintillating layers and help to distinguish high energy photons from neutral pions. The latter decays into photon pairs which may mimic high energy photons in this part of the detector with an increased likelihood. The increased granularity of the preshower detector helps mitigate this effect. The energy resolution of the ECAl is $\sim$ 1-4\%.

\begin{figure}
    \centering
    \includegraphics[width=0.9\textwidth]{figures/CMSEcal.png}
    \caption{An overview of the CMS ECAL \cite{CMSECALFigure}, shown in the r(y)-z plane. The dashed lines denote the coverage of the barrel and endcap ECAL region as well as the preshower detector.}
    \label{fig:CMSTracker}
\end{figure}

\subsection{The hadronic calorimeter}
Following the ECAl is the hadronic calorimeter (HCAL) \cite{CMSHCAL}. It is designed to measure the presence and energy of hadrons, which typically traverse the ECAL with minor energy losses. It is the most hermetic part of the CMS detector, with a coverage out to $\mid$$\eta$$\mid$ = 5.0, in order to absord almost all collision particles. The only exceptions to this are muons which are particles that minimally deposit their energy and neutrinos, which have an interaction probability that is so low that they cannot be measured with the CMS detector at all. \\
\\
In contrast to the ECAL, the HCAL is a sampling calorimeter. This means layers of absorber are interleaved with layers of a scintillator. Different materials are used in different parts of the calorimeter, which is split into the barrel ($\mid$$\eta$$\mid$ $<$ 1.5), endcap (1.5 $<$ $\mid$$\eta$$\mid$ $<$ 3.0) and forward (3.0 $<$ $\mid$$\eta$$\mid$ $<$ 5.0) regions. Since the HCAL component inside the magnet system does not sufficiently absorb all hadronic showers, the system also extends past the magnet. Due to the sampling nature of the calorimeter, a lower nuber of respective interaction lengths and larger energy fluctuations in hadronic particle showers, the energy resolution of the HCAL is significantly poorer than the ECAL. It lies in the order of 10-30$\%$ and with a strong dependence on the energy and pseudorapidity of the initiating particles.
\subsection{The superconducting solenoid magnet}
\label{subsection:CMSSolenoidMagnet}
A key component of the CMS detector is the superconducting solenoid magnet \cite{CMSMagnet}. It is responsible for maintaining a strong 3.8 T magnetic field that homogenously permeats the barrel of the detector. A measurement of the field strength can be seen in \autoref{fig:CMSMagnet}. With its toroidal shape, the field is orientated along the z-axis and covers the 12.9m barrel region of the detector, curving the the trajectories of charged particles emerging from the interaction point in the $\phi$-direction. This allows for a measurement of the particles transverse momentum $p_{T}$, which together with the $\phi$ and $\eta$ directions fully characterise the particle's momentum vector. The magnet itself is composed of of superconducting noibium-titanum coils that are cooled to a temperature of 4.65K, at which these are superconducting. The magnet is encased by a 12,000t steel yoke that captures the magentic field that is produced outside of the solenoid. 

\begin{figure}
    \centering
    \includegraphics[width=0.9\textwidth]{figures/CMSMagnet.png}
    \caption{An overview of the magnetic flux (left) and magnetic field lines(right) inside the CMS detector, shown in the r-z plane \cite{CMSMagnetFigure}.}
    \label{fig:CMSMagnet}
\end{figure}

\subsection{The muon chambers}
The muon subdetector consists of a dedicated system of gaseous detectors \cite{CMSMuonDetector}\cite{CMSMuonDetectorPerformance}, which are placed outside of the solenoid magnet. As suggested by the CMS name, a strong focus is placed on the performance of this subdetector. This is as muons may often be produced in collisions that are of physics interest (such as in this work) and thus an emphasis is laid on detecting these with great efficiency. Due to muons being minimally ionising particles, they easily pass through the inner subdetector layers to reach the muon chambers and information from the moun chambers as well as the tracker and calorimeters may be used to identify and reconstruct them.\\
\\
Like the other subdetectors, the muon chambers are separated in a barrel ($\mid$$\eta$$\mid$ $<$ 1.2) and endcap (1.2 $<$ $\mid$$\eta$$\mid$ $<$ 2.4) region, which are composed of drift tubes and cathode strip chambers respectively. The drift tubes each consists of a gas volume containing a mixture of Argon and CO$_2$ in which a posititively charged wire is stretched through the center. When charged particles such as muons traverse these tubes, the gas is ionised. Due to the positive charge of the wire, the resulting electrons drift towards the wire producing an electrical signal. Thus the presence of muons may be determined by activation of the drift tubes. The cathode strip chambers on the other hand consist of layers of positively charged (anode) wires, which are arranged in a perpendicular fashion to a set of negatively charged (cathode) strips. Combining signals from both the wires and strips allows for a position measurement in both the R and $\phi$ direction. Both types of detector are supplemented by resistive plate chambers, which act as a trigger providing a precise timing resolution of $\sim$1ns. This makes it possible to unambiguously assign muons to individual collisions. These consist of parallel, oppositely charged plastic plates that are coated with a conductive graphite layer and are contained in a gas volume. Ionisation of the gas due to the traversal of a charged particle thus leads to an electrical signal. An overview of the spetial arrangement of these systems can be see in \autoref{fig:CMSMuonSystem}. With this system, the bulk of muons may be measured with a precise momentum resolution of $\sim$1-2\%. 

\begin{figure}
    \centering
    \includegraphics[width=0.9\textwidth]{figures/CMSMuonSystem.png}
    \caption{An overview of the CMS muon system, shown in the r-z plane \cite{CMSDesignReportVol1}. Shown are the drift tube (DT), the cathode strip chambers (CSC) and resistive plate chambers (RPC).}
    \label{fig:CMSMuonSystem}
\end{figure}

\subsection{The triggering system}
The triggering system is an essential component in manging the data output of the CMS detector \cite{CMSTrigger}. With a nominal collision rate of $\sim$40 MHz, the data rate the CMS detector provides is close to 40 TB/s. Not only is the storage of such a quantity of data unfeasible but a significant portion consists of low-energy scattering events which are not of interest. As such, the triggering system is implemented to extract a subset of events that are of physics interest. \\
\\
The trigger systems is composed of two subsystems. The first the so-called level one (L1) trigger. This is a very fast hardware-based system which reduces the event rate to $\sim$100 kHz by evaluating the presence of e.g. energetic muons or other interesting signatures such as large energy deposits in the calorimeters in an event. The total time allocated to decide whether an event should be kept is 3.2$\mu$s. Subtracting for signal propagation in the detector, the L1 system must make a decision within ~1$\mu$s. From the L1, the events are passed to a software based high-level trigger (HLT) system. This is composed of several thousand CPU cores, performing a simple reconstruction of the event signatures to make a decision whether an event should be stored. Since different analyses have different needs, a set of trigger paths are defined so that only one such path must be satisfied for an event to pass the HLT. Since the HLT is software-based, the trigger paths may be continuously updated. After the HLT, the event rate is thus reduced to $\sim$100 Hz and the passing events are permanently stored.

\section{Event reconstruction with the CMS detector}
Events that pass the triggering system are stored and reconstructed using a more complicated set of reconstruction algorithms. An overview of the reconstruction techniques for the objects relevant to this work, namely muons and jets, is given in this section. 

\subsection{Track and vertex reconstruction}
Particle tracks, describing the trajectories of particles through the detector, can be obtained by leveraging information from the pixel and strip detectors of the tracker \cite{CMSTrackReco}. By determining the track of a charged particle and thus the curvature of its trajectory in the detectors magnetic field, the particle's transverse momentum \pt \, may be implicitly determined. Since track reconstruction is a computationally intensive procedure given the large number of permutations in which individual pixel or strip hits may be combined, this procedure is performed iteratively. Inititally, tracks which are easily identifiable due to e.g. their relatively high \pt \, or proximity to the interaction point are identified by matching hits in the pixel and silicon strip subdetectors and performing a fitting procedure. The hits associated with these tracks are then removed from the collection of unassociated hits. This procedure is repeated anew with looser fitting criteria so that hits that may originate from low \pt \, tracks or those with an origin displaced from the collision point, may also be associated to tracks. \\
\\
From the reconstructed tracks, common track origins or \textit{vertices} may be identified. Since several proton-proton collisions may occur in a single bunch crossing, this amounts to identifying the location of the individual collisions in an event. Tracks with a low perpendicular distance or low \textit{impact parameter} to the center of the bunch crossing and that satisfy requirements on the number of pixel and strip detector hits as well as the quality of the track fit are chosen for this purpose. These tracks are clustered using a deterministic annealing algorithm \cite{Annealing}, thus producing a set of candidate vertices with some location along the z-axis. The vertex candidate which is associated with the highest $\sum$\pt$^2$ is assigned as the primary vertex of the collision. The remaining vertex candidates are referred to as pile-up vertices. 

\subsection{The Particle Flow algorithm}
The Particle Flow (PF) algorithm \cite{ParticleFlowCMS} is used to combine information from many of the different CMS subsystems to give an improved and holistic description of an event. This includes reconstructed tracks, the energy deposits in the ECAL and HCAL as well as hits in the muon chamber system. Since different types of particles will interact with the CMS subdetector systems in unique ways, the properties of individual particles can be extrapolated from this information. These can be briefly summarised as: 

\begin{itemize}
    \item \textbf{Muons}: Muons produce clear tracks in the tracker as well as the muon system but deposit minimal amounts of energy in the calorimeters. Thus, reconstructed tracks in the tracker and muon systems can be combined to produce muon candidates. \\
    \item \textbf{Electrons}: Electrons can be identified through the fact that they produce a track in the tracker as well as energy deposits in the ECAL. However, they deposit little to no energy in the HCAL. Reconstructed tracks that can be associated to a deposit in the ECAL with no corresponding HCAL deposit thus represent electron candidates. \\
    \item \textbf{Photons}: Photons, being uncharged, do not produce a track in the tracker and only produce a signature in the ECAL. Thus ECAL deposits to which no reconstructed track or HCAL deposit can be associated are considered photon candidates. \\
    \item \textbf{Charged Hadrons}: Charged hadrons produce a track in the tracker and otherwise primarily deposit their energy in the HCAL. Thus HCAL deposits that can be matched to a reconstructed track can be considered charged hadron candidates. \\
    \item \textbf{Neutral Hadrons}: Neutral hadrons produce no track in the tracker and primarily deposit their energy in the HCAL. Thus HCAL deposits that cannot be matched to a reconstructed track can be considered neutral hadron candidates \\
\end{itemize}
A visual overview of these signatures and the particle type they correspond to can be found in \autoref{fig:CMSPF}. The PF algorithm leverages exactly these properties. Initially, matched tracks in the tracker and muon systems are identified as muons and the corresponding components are removed from the event. Subsequently, matched tracks and energy deposit clusters in the ECAL are identified as electrons and the corresponding components are removed.  An isolated cluster in the ECAL with no associated track is reconstructed as a photon candidate and the corrsponding cluster is removed. This is expected to leave only charged and neutral hadrons. Clusters of energy deposits in the HCAL associated with a track are thus identified as charged hadrons. However, it frequently occurs that photons are produced in the decay of neutral hadrons. Thus, if the energy estimated from a track is considerably less than the associated cluster in the HCAL and there is a corresponding energy deposit in the ECAL, an additional photon candidate is reconstructed that is assocaited with the hadron. Finally, HCAL clusters with no associated track and reconstructed as neutral hadrons. \\
\\
The following section describe in greater detail the reconstruction of objects relevant to this work. This includes muons, \textit{jets}, which are collimated particle showers that typically consist of a collection of reconstructed objects and missing transverse energy. 

\begin{figure}
    \centering
    \includegraphics[width=0.9\textwidth]{figures/PF.pdf}
    \caption{An transverse slice of the CMS detector, visualising the signatures that different particles produce in the different detector subsystems. \cite{ParticleFlowCMS}. }
    \label{fig:CMSPF}
\end{figure}

\subsection{Reconstruction and identification of muons}

\subsection{Reconstruction and identification of jets}

\subsection{Missing transverse energy}